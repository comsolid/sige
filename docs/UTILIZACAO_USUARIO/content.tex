\section{Manual do Usuário}

Este manual serve para o administrador do SiGE entender como funciona as
telas do usuário. Pretendemos mostrar cada tela e identificar sua
função.

\subsection{Telas de acesso de usuários deslogados}

São telas de acesso público, onde qualquer usuário pode acessar.

\subsubsection{Página inicial \label{home}}

Mapeada como: \texttt{/ {[}application/IndexController.index{]}}

Primeira página do sistema, contendo o banner, links para a inscrição
\ref{inscricao} e login \ref{login}, além de contador regressivo para o
evento.

\subsubsection{Login \label{login}}

Mapeada como: \texttt{/login {[}application/IndexController.login{]}}

Página comum de login, onde o e-mail e a senha são requeridos. Contém
links para Esqueci minha senha \ref{esqueci-senha} e inscrição
\ref{inscricao}.

\subsubsection{Inscrição \label{inscricao}}

Mapeada como :
\texttt{/participar {[}application/ParticipanteController.criar{]}}

Página para preenchimento dos dados necessário para se cadastrar. Logo
após o cadastro um e-mail é enviado para o usuário contendo alguns dados
dele, além de uma senha gerada automaticamente pelo sistema.

\paragraph{OBSERVAÇÃO:}

esse sistema de enviar a senha deve mudar em breve, pois isso pode
causar um problema de segurança.

\subsubsection{Esqueci minha senha \label{esqueci-senha}}

Mapeada como:
\texttt{/recuperar-senha {[}application/IndexController.recuperar-senha{]}}

Nessa tela apenas o e-mail é requerido. Logo em seguida uma nova senha
gerada pelo sistema é enviada para o e-mail do usuário.

\paragraph{OBSERVAÇÃO:}

essa tela deve mudar muito. Pedir apenas o e-mail pode gerar ataque de
DoS, troca de senhas não desejadas, etc.

Para começar teremos um Captcha, além do pedido de outros dados.

\subsubsection{Sobre \label{sobre}}

Mapeada como: \texttt{/sobre {[}application/IndexController.sobre{]}}

Página com os envolvidos no projeto, tecnologias utilizadas.

\subsubsection{Programação \label{programacao}}

Mapeada como:
\texttt{/programacao {[}application/EventoController.programacao{]}}

Página com toda a programação do evento, com detalhes sobre o
palestrante, horário, conteúdo abordado, etc.

\subsubsection{Detalhes do evento \label{evento}}

Página com todos os detalhes do evento, se é palestra, mini-curso ou
oficina, o horário, nível, links para compartilhar em redes sociais e
comentários usando a plataforma \href{http://disqus.com/}{disqus}.

\paragraph{NOTA:}

usamos a palavra \textbf{Encontro} para definir um conjunto de eventos.
\textbf{Evento} é definido como uma palestra, mini-curos ou oficina.

\subsubsection{Página do usuário \label{usuario}}

Cada usuário tem uma página pública onde é mostrado alguns detalhes
sobre ele como bio, twitter, apresentações do slideshare.

\paragraph{OBSERVAÇÃO:}

devido a uma mudança no RSS do slideshare, alguns usuários apresentam
problemas com a visualização das suas apresentações mesmo colocando o
usuário do slideshare corretamente. Para isso precisamos usar a nova
\href{http://apiexplorer.slideshare.net/}{api}.

\subsection{Telas de acesso de usuários logados}

