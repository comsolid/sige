\section{Manual do Administrador}

Este manual serve para o administrador do SiGE entender como funciona as
telas do administrador. Pretendemos mostrar cada tela e identificar sua
função.

Um administrador e um usuário comum são diferenciados apenas por uma
coluna no banco de dados, na tabela \texttt{pessoa} a coluna
\texttt{administrador} com o valor \texttt{true}.

É possível tornar qualquer usuário administrador através do SiGE, como
veremos em \ref{permissao-usuario}.

Ao se logar o administrador ver a mesma página do usuário com a
diferença que no menu principal aparece o item \textbf{Admin}.

\subsubsection{Inscrições \label{inscricoes}}

Mapeada como:
\texttt{/admin ou /inscricoes {[}admin/ParticipanteController.index{]}}

Ao clicar no item \textbf{Admin} do menu principal você é levado para a
página de incrições. Aqui é onde você, no dia do evento, faz a
confirmação dos participantes que se inscreveram previamente. Com isso
você confirma que aquele participante realmente esteve no Encontro e
assim o certificado dele fica habilitado para download.

Na lista aparecerão somente os 20 primeiros cadastrados no Encontro.
Então é preferível que você utilize a busca, para ter resultado
precisos.

Ao encontrar a pessoa basta clicar em \textbf{Confirmar}. Caso tenha se
engadado clique em \textbf{Desfazer}. Tudo é feito sem que a página seja
recarregada, basta esperar que a palavra Confirmar se torne Desfazer.
Além disso uma mensagem vai aparecer confirmando o pedido.

Ou seja, essa é a página principal no dia do Encontro, então para as
pessoas da recepção essa é a página que você deve mostrar e ensinar os
macetes para um cadastramento rápido, evitanto filas muito grandes.

\paragraph{DICA 1:}

Ao chegar escolas ou instituições utilize a Busca por Instituição, como
ela traz até 20 pessoas é bem provável que venham todos os alunos e
poupa seu trabalho de pesquisar nome por nome.

\paragraph{DICA 2:}

Faça o mesmo para a Caravana, busque pela Caravana e tenha todos os
participantes daquela caravana na lista.

\subsubsection{Caravana \label{caravana}}

Mapeada como:
\texttt{/admin/caravana {[}admin/CaravanaController.index{]}}

Nessa tela é possível ver as caravanas que desejam participar do evento,
com os dados da caravana, responsável, quantidade de homens e mulheres.

Nessa tela ainda existe o botão validar e invalidar, mas na realidade o
melhor é conversar com o responsável e definir com ele os detalhes da
vinda da caravana. Essa tela deve servir mesmo como controle, para você
administrador junto com sua equipe decidir se é possível ou não a vinda
da caravana.

\subsubsection{Evento \label{evento}}

Mapeada como: \texttt{/admin/evento {[}admin/EventoController.index{]}}

Essa tela é essencial principalmente antes e depois do encontro.

\paragraph{DEFINIÇÃO:}

Definimos \textbf{Encontro} com sendo um conjunto de \textbf{Eventos},
onde os \textbf{Eventos} são definidos como \textbf{Palestras},
\textbf{Mini-cursos} ou \textbf{Oficinas}.

\bigskip

Antes do encontro, essa é a tela onde você irá acompanhar a submisão de
trabalhos. Aqui você sabe os dados do evento, junto com um link para os
detalhes do evento \ref{detalhes-evento}.

\subsubsection{Detalhes do Evento \label{detalhes-evento}}

Mapeada como:
\texttt{/admin/evento/detalhes/id/:id {[}admin/EventoController.detalhes{]}}

Nessa tela além dos dados mostrados na tela de Eventos \ref{evento},
você pode ver um Resumo, Preferência de Horário e as Tecnologias
envolvidas. Esse último item é muito útil pois ele diz se o usuário
precisará de ferramentas ou aparelhos específicos para sua palestra.
Exemplo: Internet na sala, Entrada para HDMI, Aparelho de Som, plugin do
VLC, Ambiente Python, enfim.

Logo abaixo você tem a lista de horários, onde é possível Criar um
horário, Editar \ref{salvar-horario} ou Deletar. Esse trecho define os
horários inicial e final do evento além da sala e data que acontecerá.

Mais abaixo tem a lista de outros palestrantes.

Após criar os horários é hora de validar o evento. Para isso basta
clicar em \textbf{Validar} no sub menu. Também é possível editar os
dados do evento clicando em \textbf{Editar}.

Ao final do evento é preciso que você confirme quem realmente palestrou.
Para isso basta clicar em \textbf{Apresentado} no sub menu. Isso faz com
que o certificado de palestrante apareça para o participante.

\subsubsection{Criar/Editar Horário \label{salvar-horario}}

Mapeada como:
\texttt{/admin/horario/criar/evento/:evento {[}admin/HorarioController.criar{]}}

\texttt{/admin/horario/editar/evento/:evento/id/:id {[}admin/HorarioController.editar{]}}

Tela para adicionar ou editar o horário do evento, cada evento pode ter
um ou mais horários. Algumas oficinas e mini-cursos às vezes levam mais
de um dia para serem realizados. Ou uma palestra pode ser muito
requisitada e pode ser apresentada mais de uma vez.

Caso haja choque de horário na mesma sala, no mesmo dia, o SiGE te
avisa.

Logo abaixo aparece os horários já cadastrados.

Os horários são de uma em uma hora, ao selecionar o horário inicial o
horário de término é atualizado para uma hora a mais.

\paragraph{OBSERVAÇÃO:}

A configuração do intervalo de horas e a hora mínima e máxima atualmente
estão no código-fonte, ou seja, para modificar seu comportamento é
preciso alterar o código. Em uma outra versão do SiGE isso pode mudar.
Mas enquanto isso o código-fonte para essa mudança pode ser encontrado
em: \texttt{application/forms/Horarios.php} na linha \texttt{77}.

\subsubsection{Relatórios}

Essa tela é auto explicativa :)

\subsubsection{Configurações}

Nessa tela tem o Gerenciamento dos Encontros \ref{gerenciar-encontro} e
Permissão de usuários \ref{permissao-usuario}.

\subsubsection{Gerenciamento dos Encontros \label{gerenciar-encontro}}

Para criar o primeiro encontro é preciso usar diretamente o banco de
dados, mas nos encontros seguintes use o SiGE. Nessa tela é possível ver
a lista de encontros, Criar ou Editar encontro, Editar mensagens de
e-mail.

\subsubsection{Permissão de usuários \label{permissao-usuario}}

Essa tela serve para procurar participantes e torná-los administradores
ou usuários comuns, e ainda definir se ele é da Organização ou
Coordenação do encontro. Essa última opção ainda não influei no sistema
mas para as próximas versões podemos usar essa informação.
